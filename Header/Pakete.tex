% Codierung dieser Datei ist UTF8
\usepackage[utf8]{inputenc}

% Deutsches Sprachpaket
\usepackage[ngerman]{babel}

% Paket für quotes
\usepackage[babel, german=quotes]{csquotes}

% Link Packet
% Links werden mit dunklen Blau markiert, statt mit Boxen
\usepackage{hyperref}
\usepackage{xcolor}
\hypersetup{
	colorlinks,
	linkcolor={blue!50!black},
	citecolor={blue!50!black},
	urlcolor={blue!80!black}
}

% Fix um zu Bild statt Unterschrift zu springen
\usepackage[all]{hypcap}

% images
\usepackage{graphicx}

% source code listing
\usepackage{listings}

% Farben für Source code
\usepackage{color}
\definecolor{mygreen}{rgb}{0,0.6,0}
\definecolor{mygray}{rgb}{0.5,0.5,0.5}
\definecolor{mymauve}{rgb}{0.58,0,0.82}

% Einstellungen für Codeblocks
\lstset{ 
  backgroundcolor=\color{white},   % choose the background color; you must add \usepackage{color} or \usepackage{xcolor}; should come as last argument
  basicstyle=\footnotesize,        % the size of the fonts that are used for the code
  breakatwhitespace=false,         % sets if automatic breaks should only happen at whitespace
  breaklines=true,                 % sets automatic line breaking
  captionpos=b,                    % sets the caption-position to bottom
  commentstyle=\color{mygreen},    % comment style
  deletekeywords={...},            % if you want to delete keywords from the given language
  escapeinside={\%*}{*)},          % if you want to add LaTeX within your code
  extendedchars=true,              % lets you use non-ASCII characters; for 8-bits encodings only, does not work with UTF-8
  frame=single,	                   % adds a frame around the code
  keepspaces=true,                 % keeps spaces in text, useful for keeping indentation of code (possibly needs columns=flexible)
  keywordstyle=\color{blue},       % keyword style
  language=Octave,                 % the language of the code
  morekeywords={*,...},            % if you want to add more keywords to the set
  %numbers=left,                    % where to put the line-numbers; possible values are (none, left, right)
  numbersep=5pt,                   % how far the line-numbers are from the code
  numberstyle=\tiny\color{mygray}, % the style that is used for the line-numbers
  rulecolor=\color{black},         % if not set, the frame-color may be changed on line-breaks within not-black text (e.g. comments (green here))
  showspaces=false,                % show spaces everywhere adding particular underscores; it overrides 'showstringspaces'
  showstringspaces=false,          % underline spaces within strings only
  showtabs=false,                  % show tabs within strings adding particular underscores
  stepnumber=1,                    % the step between two line-numbers. If it's 1, each line will be numbered
  stringstyle=\color{mymauve},     % string literal style
  tabsize=2,	                   % sets default tabsize to 2 spaces
}

%Workaround für Sonderzeichen in Code
\lstset{literate=%
	{Ö}{{\"O}}1
	{Ä}{{\"A}}1
	{Ü}{{\"U}}1
	{ß}{{\ss}}1
	{ü}{{\"u}}1
	{ä}{{\"a}}1
	{ö}{{\"o}}1
	{~}{{\textasciitilde}}1
}

% Ersetze "Listing 1" durch "Anhang 1"
%\renewcommand{\lstlistingname}{Anhang}

% erhöhen sorgt für öftere Zeilenumrüche durch Wörter
\tolerance=1200

% Biblatex für Literatur mit biber Backend
\usepackage[
	backend=biber, 
	style=alphabetic,
	sortlocale=de_DE
	citestyle=authoryear,
	maxbibnames=99 % Alle Autoren einer Quelle auflisten
]{biblatex}
\bibliography{Literatur/Literatur}

% Glossar
\usepackage[toc]{glossaries}

% Keine Einrückung
\setlength{\parindent}{0pt}

\usepackage{multicol}

\usepackage{pdfpages}

% force image location
\usepackage{float}

\usepackage{tabularx}

% used by pandas generated tables
\usepackage{booktabs}

% Packet für Definitionen
\usepackage{amsthm}
\newtheoremstyle{definitionstyle}% name of the style to be used
	{7pt}% measure of space to leave above the theorem. E.g.: 3pt
	{7pt}% measure of space to leave below the theorem. E.g.: 3pt
	{}% name of font to use in the body of the theorem
	{}% measure of space to indent
	{}% name of head font
	{:}% punctuation between head and body
	{ }% space after theorem head; " " = normal interword space
	{\textbf{\thmname{#1} \thmnote{#3}}}% Manually specify head

% verwendet definierten Style
\theoremstyle{definitionstyle}

% Umgebunf für Definitionen
\newtheorem*{definition}{Definition}
